\inputDefaultByFormat{layout}

\usepackage{hyperref}
\usepackage{amsmath}
\usepackage{amssymb}
\usepackage{booktabs}
\usepackage[pdftex]{graphicx}
\usepackage{hyperref}
\usepackage{ngerman}
\usepackage[utf8]{inputenc}
\usepackage{tabularx}
\usepackage[toc]{glossaries}
\usepackage{float}
\usepackage{multirow}
\usepackage{blindtext}
\usepackage{lipsum}
\usepackage{xargs}
\usepackage[pdftex,dvipsnames]{xcolor}
\usepackage[colorinlistoftodos,prependcaption,textsize=tiny]{todonotes}
\usepackage{xcolor}
\usepackage{listings}
\usepackage{picins}
\usepackage[autostyle]{csquotes} 

\definecolor{backcolour}{rgb}{0.95,0.95,0.92}

\lstset{ 
    basicstyle=\footnotesize\ttfamily,
    keywordstyle=\bfseries\color{BurntOrange},
	commentstyle=\itshape\color{gray},
    numbers=left, % where to put the line-numbers
    numberstyle=\tiny, % the size of the fonts that are used for the line-numbers     
    backgroundcolor=\color{backcolour},
    showspaces=false, % show spaces adding particular underscores
    showstringspaces=false, % underline spaces within strings
    showtabs=false, % show tabs within strings adding particular underscores
    frame=single, % adds a frame around the code
    tabsize=2, % sets default tabsize to 2 spaces
    rulesepcolor=\color{gray},
    rulecolor=\color{black},
    captionpos=b, % sets the caption-position to bottom
    breaklines=true, % sets automatic line breaking
    breakatwhitespace=false,
	identifierstyle=\color{blue},
	stringstyle=\color{OliveGreen},
	xleftmargin=5.0ex
}


\usepackage{tikz}
\usetikzlibrary{matrix,chains,positioning,decorations.pathreplacing,arrows}

\makeglossaries
\loadglsentries[main]{../default/glossar}

% http://tex.stackexchange.com/questions/9796/how-to-add-todo-notes
\newcommandx{\unsure}[2][1=]{\todo[linecolor=red,backgroundcolor=red!25,bordercolor=red,#1]{#2}}
\newcommandx{\change}[2][1=]{\todo[linecolor=blue,backgroundcolor=blue!25,bordercolor=blue,#1]{#2}}
\newcommandx{\info}[2][1=]{\todo[linecolor=OliveGreen,backgroundcolor=OliveGreen!25,bordercolor=OliveGreen,#1]{#2}}
\newcommandx{\improvement}[2][1=]{\todo[linecolor=Plum,backgroundcolor=Plum!25,bordercolor=Plum,#1]{#2}}
\newcommandx{\thiswillnotshow}[2][1=]{\todo[disable,#1]{#2}}
