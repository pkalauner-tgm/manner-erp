\section{Konkurrenz}
\subsection{Ferrero}
Ferrero ist ein italienischer Süßwarenhersteller, welcher international, inklusive Österreich, mit 21 Produktionsstätten tätig ist. Das Unternehmen beschäftigt ca. 34.236 Mitarbeiter und erwirtschaftete 2014 einen Umsatz von 8,4 Milliarden Euro. \cite{wiki_ferrero} \\\\
Mit Produkten wie Duplo, der Milchschnitte und Nutella und dem generell breiteren Produktspektrum ist Ferrero ein durchaus ernstzunehmender Konkurrent für Manner.
\subsection{Bahlsen}
Das deutsche Familienunternehmen mit Sitz in Hannover wurde 1889 von Hermann Bahlsen gegründet. Bahlsen beschäftigt 2537 Mitarbeiter (Stand 2013) und erwirtschaftete 2013 526 Mio. Euro. \cite{wiki_bahlsen}\\\\
Bahlsen ist Marktführer in Deutschland und europaweit einer der führenden Anbieter von Süßgebäck. Das Unternehmen besitzt ebenfalls die nationalen Marken Kornland (Österreich), Krakuski (Polen) und Brandt (Deutschland). Die Produkte werden an fünf europäischen Standorten produziert und in mehr als 80 Länder exportiert. \cite{bahlsen}
\subsection{S. Spitz GmbH}
Spitz ist ein großer österreichischer Nahrungsmittelhersteller. Der Firmensitz liegt in Attnang-Puchheim, Oberösterreich. Spitz produziert hauptsächlich für den österreichischen Markt, ein Drittel der Produkte wird exportiert. Im Jahr 2005 betrug die Mitarbeiteranzahl 650 und der Umsatz 200 Mio. Euro. \cite{wiki_spitz} \\\\
Neben alkoholfreien Getränken und Spirituosen, bietet Spitz auch Backwaren, darunter auch sechs verschiedene Waffelsorten, an. \cite{spitz_produkte}