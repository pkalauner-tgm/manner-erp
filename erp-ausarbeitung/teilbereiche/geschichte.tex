\section{Geschichte}
Josef Manner I. eröffnete 1890 ein kleines Geschäft in der nähe des Stephansdoms, wo er Tafelschokolade und Feigenkaffee verkaufte. Doch da ihm die Qualität der Schokolade seiner Lieferanten nicht gut genug war, kaufte er ein Lokal eines kleinen Schokoladenerzeugers im fünften Wiener Gemeindebezirk. Daher war er ab 1. März 1890 Inhaber der ''Chocoladenfabrik Josef Manner''. \\
Zu Beginn war Josef Manner I. noch Erzeuger, Verkäufer und Werbeagent in einer Person und lieferte seine Waren oft selbst an die Kunden. Doch bereits im Gründungsjahr musste er expandieren. Bis 1897 trug er bereits die Verantwortung für über 100 Mitarbeiter und der Firmensitz wurde in die Hernalser Kulmgasse verlagert und die Firma ''Chocolade Manner'' wurde gegründet.\\
Innerhalb eines Jahrzehnts wurde die Firma zu einen der führenden Süßwarenunternehmen der österreichisch-ungarischen Monarchie. 
Die Manner-Schnitte findet sich zum ersten Mal 1898 in einem Sortimentskatalog des Hauses Manner, und zwar unter dem eher sachlichen Namen ''Neapolitaner Schnitte No. 239''. \\
Am 16. Oktober 1900 verkaufte Josef Manners damaliger Partner seinen Anteil der Firma an Johann Riedl  und legte so den Grundstein für die bis heute währende Zusammenarbeit der Familien Manner und Riedl.
Am 23. Oktober 1913 wandelte man die Firma Josef Manner \& Co. in eine Aktiengesellschaft mit dem Namen Josef Manner \& Comp AG um. Zu dieser Zeit umfasste das Unternehmen bereits 3000 Mitarbeiter und verfügte über einen Fuhrpark mit 60 Pferden und mittlerweile auch über Produktionsanlagen auf dem modernsten Stand der damaligen Zeit. \\
Während der ersten Weltkriegs konnte sich die Firma nur mit mühen über Wasser halten, da der Arbeitsmarkt der Donaumonarchie von 56 Millionen auf die gerade noch sechs Millionen Einwohner der Ersten Republik Österreich schrumpfte. Der zweite Weltkrieg brachte die Firma um ihre letzten Vorräte.\\
Am 5. Mai 1947 verstarb der Firmengründer Josef Manner I. Doch Ende der 40. Jahre wurde das Unternehmen wieder sehr schnell und vor allem International Erfolgreich. 1960 gelang der Firma der Absprung ins Technologie-Zeitalter. Die neuartige Verpackung garantierte nicht nur eine längere Haltbarkeit, sondern auch ein leichtes Öffnen der Packung. 1964 wurde erstmals seit 1914 der Rekordumsatz überschritten werden.\\
1970 erfolgte der Zusammenschluss mit dem zweitgrößten österreichischen Süßwarenunternehmen, der Firma Napoli, Ragendorfer \& Co. 1996 wurde die Firma Walde Candita in Wolkersdorf/NÖ übernommen. 2000 feierte schließlich auch die Firma Victor Schmidt \& Söhne GmbH mit den Marken „Ildefonso“ und „Victor Schmidt Austria Mozartkugeln“ ihren Einstand in der Manner-Großfamilie. \cite{josef_manner_unternehmen}